\documentclass{article}

\usepackage {amsmath}
\usepackage {slashed}

\begin {document}
\section{Di-lepton production in $e^+e^-$ in the SM}

\subsection{QED}
Using the Feynman rules for scalar QED we get an amplitude 
%
\begin{equation}
  i\mathcal{M} =%
		\bar{v}^{s'}(p')(-ie\gamma^{\mu})u^s(p)%
		\left(\frac{-ig_{\mu\nu}}{q^2}\right)%
		\bar{u}^r(k)(-ie\gamma^{\nu})v^{r'}(k')
\end{equation}
%
Where $p$ and $p'$ are the momentum for the incoming $e^+e^-$,%
and $k$ and $k'$ are the momentums of the outgoing $\tau^+\tau^-$.
$s$ and $r$ are the spin indicies, $q$ is the momentum of the force 
mediator and $e$ is the elementary charge.
On a more compact form leaving the spin superscripts implicit
%
\begin{equation}
  i\mathcal{M} =%
		\frac{ie^2}{q^2}\left(\bar{v}(p')\gamma^{\mu}u(p)\right)%
		\left(\bar{u}(k)\gamma_{\mu}v(k')\right)
\end{equation}
%
Then using $(\bar{v}\gamma^{\mu}u)^* = \bar{u}\gamma^{\mu}v$ 
to get the $|\mathcal{M}|^2$ leads to
%
\begin{equation}
|\mathcal{M}|^2 = \frac{e^4}{q^4}%
		  \Big(\bar{v}(p')\gamma^{\mu}u(p)\bar{u}(p)\gamma^{\nu}v(p')\Big)%
		  \Big(\bar{u}(k)\gamma_{\mu}v(k')\bar{v}(k')\gamma_{\nu}u(k)\Big)
\label{sqM}
\end{equation}
%
Now taking the spins into account we have an expression on the form
%
\begin{equation}
  \frac{1}{4}\sum_s\sum_{s'}\sum_r\sum_{r'}|\mathcal{M}(s,s'\rightarrow r,r')|^2.
\end{equation}
%
With the 2 completeness relations
%
\begin{equation}
  \sum_s u^s(p)\bar{u}^s(p) = \slashed{p} + m\,;\quad\quad%
  \sum_s v^s(p)\bar{v}^s(p) = \slashed{p} - m
  \label{comprel}
\end{equation}
%
used in the the first parenthesis in equation (\ref{sqM}) written out in spinor indicies,
making it possible to move $v$ and $\bar{v}$ next to each other we get
\begin{flalign*}
  \sum_{s,s'} \bar{v}^{s'}_{a}(p')\gamma^{\mu}_{ab}u^s_b(p)%
  \bar{u}^s_c(p)\gamma^{\nu}_{cd}v^{s'}_d(p') &=%
  (\slashed{p}'-m)_{da}\gamma^{\mu}_{ab}(\slashed{p} + m)_{bc}\gamma^{\nu}_{cd}\\
  &= \text{tr}[(\slashed{p}' -m)\gamma^{\mu}(\slashed{p}+m)\gamma^{\nu}]
\end{flalign*}
thus our squared matrix element becomes the product of 2 traces
%
\begin{equation}
  \frac{1}{4}\sum_{\text{spins}}|\mathcal{M}|^2 = %
  \frac{e^4}{4q^4}%
  \text{tr}\left[(\slashed{p}' - m_e)\gamma^{\mu}(\slashed{p} + m_e)\gamma^{\nu}\right]%
  \text{tr}\left[(\slashed{k} + m_{\tau})\gamma_{\mu}(\slashed{k}' - m_{\tau})\gamma_{\nu}\right]%
\end{equation}
%
Insert trace technology here
%
Using the trace relations we get for the $e$ part
%
\begin{equation}
  \text{tr}\left[(\slashed{p}' - m_e)\gamma^{\mu}(\slashed{p} + m_e)\gamma^{\nu}\right] =%
  4\left[p^{'\mu}p^{\nu} + p^{'\nu}p^{\mu} - g^{\mu\nu}(p\cdot p' + m_e^2)\right]
\end{equation}
and for the $\tau$
\begin{equation}
  \text{tr}\left[(\slashed{k} + m_{\tau})\gamma_{\mu}(\slashed{k}' + m_{\tau})\gamma_{\nu}\right] =%
  4\left[k_{\mu}k'_{\nu} + k_{\nu}k'_{\mu} - g_{\mu\nu}(k\cdot k' + m_{\tau}^2)\right]
\end{equation}
%
There is a large difference in $m_e << m_{\tau}$ which makes it reasonable to set $m_e = 0$,
multiplying the traces gives us a square matrix element of
%
\begin{equation}
  \frac{1}{4}\sum_{\text{spins}}|\mathcal{M}|^2 =%
  \frac{8e^4}{q^4}%
  \left[(p\cdot k)(p'\cdot k') + (p\cdot k')(p'\cdot k) + m^2_{\tau}(p\cdot p')\right]
  \label{genEq}
\end{equation}
%
Furterhmore we choose the center of mass reference frame and translates our momentums into
kinematic variables instead, energies and angles.
%
\begin{equation}
\begin{array}{ccc}
  q^2 = (p+p')^2 = 4E^2 &\,;\quad& p\cdot p' = 2E^2\\
  p\cdot k = E^2 - E|\textbf{k}|\cos\theta &\,;\quad& p\cdot k' = E^2 + E|\textbf{k}|\cos\theta
\end{array}
\end{equation}
%
Rewriting eq. (\ref{genEq}) in terms of $E$ and $\theta$ we get
%
\begin{flalign}
  \frac{1}{4}\sum_{\text{spins}}|\mathcal{M}|^2 &=%
  \frac{8e^4}{16E^4}%
  \Big[ %
    E^2(E-|\textbf{k}|\cos\theta)^2 +%
    E^2(E+|\textbf{k}|\cos\theta)^2 +%
    2m_{\tau}^2E^2]
  \Big]\\
  %
  &= e^4%
  \left[ %
    \left(1+\frac{m^2_{\tau}}{E^2}\right) +%
    \left(1 - \frac{m^2_{\tau}}{E^2}\right)\cos^2\theta%
  \right]
\end{flalign}
%
Now that we have $|\mathcal{M}|^2$  we can put it into a formula for $d\sigma/d\cos\theta$ 
derived in Peskin\footnote{Page 107, Equation 4.84}
%
\begin{equation}
  \left(\frac{d\sigma}{d\Omega}\right)_{CM} =%
  \frac{1}{2E_A 2E_B |v_p - v_{p'}|}\frac{|\mathbf{k}|}{(2\pi)^24E_{CM}}|\mathcal{M}|^2
\end{equation}
%
In the center of mass frame the relative speed $|v_{p}- v_{p'}|$ becomes 2, and $E_p=E_{p'}=E_{CM}/2 = \sqrt{s}/2$. 
With a symmetry about the longitudinal direction we can make the differential cross section
\begin{equation}
  \frac{d\sigma}{d\cos\theta} = %
  \frac{\mathbf{|k|}}{32\pi^2 s^{3/2}} |\mathcal{M}|^2 =%
  \frac{1}{32\pi^2s}\sqrt{1-4\frac{m_{\tau}^2}{s}}|\mathcal{M}|^2
\end{equation}
%
Integrating with respect to $\cos\theta$ we aquire the total cross section. Keeping the prefactors
out of the calculation since they are not dependent on $cos\theta$ and integrate our expression
for the squared matrix element.
%
\begin{flalign*}
  \int^1_{-1}d\cos\theta e^4|\mathcal{M}|^2 &=%
  e^4\left[\left(1+4\frac{m_{\tau}^2}{s}\right)\cos\theta+%
  \frac{1}{3}\left(1-4\frac{m_{\tau}^2}{s}\right)\cos^3\theta\right]^{1}_{-1}\\
  &= \frac{8}{3}\left(1 + 2\frac{m^2_{\tau}}{s}\right)
\end{flalign*}
Combining combining this with our differential cross section we get an expression for the
total cross section
\begin{flalign*}
  \sigma &=  \frac{4}{3}\frac{e^4}{32\pi^2s}%
  \sqrt{1-4\frac{m^2_{\tau}}{s}}\left(1+2\frac{m^2_{\tau}}{s}\right)\\
  &= \frac{2}{3}\frac{\alpha^2}{s}%
  \sqrt{1-4\frac{m^2_{\tau}}{s}}\left(1+2\frac{m^2_{\tau}}{s}\right)
\end{flalign*}
with $\alpha = e^2/(4\pi)$.

\subsection{Electroweak}
This unifcation requires both the $\mathcal{M}_{QED}$ and $\mathcal{M}_{WI}$ 
matrix elements. Using the feynman-rules from the weak model our $\mathcal{M}_{WI}$ 
will take the form
%
\begin{equation}
  \mathcal{M}_{WI} = -\frac{g^2_Z}{s - m^2_Z + im_z\Gamma_Z}g_{\mu\nu}%
  \left[\bar{v}(p')\gamma^{\mu}\frac{1}{2}(c^e_V - c^e_A\gamma^5)u(p)\right]%
  \left[\bar{u}(k)\gamma^{\mu}\frac{1}{2}(c^{\tau}_V - c^{\tau}_A\gamma^5)v(k')\right]
\end{equation}
%
where $1/(s - m^2_Z + im_Z\Gamma_Z) = P_Z(s)$ is the Z propogator, $c^{e/\tau}_{V/A}$
are the vector and axial-vector couplings of the Z to our leptons.
It will be handy to rewrite those couplings to left- and right-handed
chiral states as $c_V = c_L + c_R$ and $c_A = c_L - c_R$
%
\begin{equation}
  -P_Zg_Z^2g_{\mu\nu}%
  \left[%
    c_L^e \bar{v}(p') \gamma^{\mu} P_L u(p) +%
    c_R^e \bar{v}(p') \gamma^{\mu} P_R u(p) %
  \right]%
  \left[%
    c_L^{\tau} \bar{u}(k) \gamma^{\nu} P_L v(k') +%
    c_R^{\tau} \bar{u}(k) \gamma^{\nu} P_R v(k') %
  \right]
\end{equation}
%
where $P_L$ and $P_R$ are the chiral projection operators $\frac{1}{2}(1\mp \gamma^5)$.
Using these chiral projection operators on a particle state give the result
\[ P_Lu=u_{\downarrow},\, P_Ru = u_{\uparrow},\, P_Lv = v_{\uparrow},\, P_Rv=v_{\downarrow} \]
And with helicity combinations like $\bar{u}_{\uparrow}\gamma^{\mu}v_{\uparrow}$ giving zero
matrix elements we are left with four helicity combinations
%
\begin{flalign}
  \mathcal{M}_{RR} &=-P_Z g^2_Z c^{e}_{R}c^{\tau}_{R} g^{\mu\nu}%
  \left[ \bar{v}_{\downarrow}(p')\gamma^{\mu}u_{\uparrow}(p)\right]%
  \left[ \bar{u}_{\uparrow}(k)\gamma^{\nu}v_{\downarrow}(k')\right]\\
  \mathcal{M}_{RL} &=-P_Z g^2_Z c^{e}_{R}c^{\tau}_{L} g^{\mu\nu}% 
  \left[ \bar{v}_{\downarrow}(p')\gamma^{\mu}u_{\uparrow}(p)\right]%
  \left[ \bar{u}_{\downarrow}(k)\gamma^{\nu}v_{\uparrow}(k')\right]\\
  \mathcal{M}_{LR} &=-P_Z g^2_Z c^{e}_{L}c^{\tau}_{R} g^{\mu\nu}% 
  \left[ \bar{v}_{\uparrow}(p')\gamma^{\mu}u_{\downarrow}(p)\right]%
  \left[ \bar{u}_{\uparrow}(k)\gamma^{\nu}v_{\downarrow}(k')\right]\\
  \mathcal{M}_{LL} &=-P_Z g^2_Z c^{e}_{L}c^{\tau}_{L} g^{\mu\nu}% 
  \left[ \bar{v}_{\uparrow}(p')\gamma^{\mu}u_{\downarrow}(p)\right]%
  \left[ \bar{u}_{\uparrow}(k)\gamma^{\nu}v_{\uparrow}(k')\right]
\end{flalign}
%
The combinations of these four-vector currents can be shown to come to a simpler form
as with this example
%
\[
  g_{\mu\nu} [\bar{v}_{\downarrow}(p')\gamma^{\mu}u_{\uparrow}(p)]%
  [\bar{u}_{\uparrow}(k)\gamma^{\nu}v_{\downarrow}(k')] = s(1+\cos\theta)
\]
%
using this on each helicity state we get
%
\begin{flalign}
  |\mathcal{M}_{RR}|^2 &=|P_Z|^2 g^4_Z (c^{e}_{R})^2(c^{\tau}_{R})^2 s^2(1+\cos\theta)^2\\
  |\mathcal{M}_{RL}|^2 &=|P_Z|^2 g^4_Z (c^{e}_{R})^2(c^{\tau}_{L})^2 s^2(1-\cos\theta)^2\\
  |\mathcal{M}_{LR}|^2 &=|P_Z|^2 g^4_Z (c^{e}_{L})^2(c^{\tau}_{R})^2 s^2(1-\cos\theta)^2\\ 
  |\mathcal{M}_{LL}|^2 &=|P_Z|^2 g^4_Z (c^{e}_{L})^2(c^{\tau}_{L})^2 s^2(1+\cos\theta)^2 
\end{flalign}
%
The spin average of the full matrix elemnt $|\mathcal{M}|^2$ is then the sum of each helicity 
state squared
%
\[ 
  <|\mathcal{M}|^2> = \frac{1}{4} \left( |\mathcal{M}_{RR}|^2 + |\mathcal{M}_{LL}|^2 +%
  |\mathcal{M}_{RL}|^2 + |\mathcal{M}_{LR}|^2\right)
\]
%
writing the terms out gives us
%
\begin{flalign*}
  <|\mathcal{M}|^2> = %
  \frac{1}{4}|P_Z|^2 g_Z^4 s^2%
  \Big(%
  &\left[ (c^e_R)^2(c^{\tau}_R)^2 + (c^e_L)^2(c^{\tau}_L)^2 \right](1+\cos\theta)^2 \\+%
  &\left[ (c^e_R)^2(c^{\tau}_L)^2 + (c^e_L)^2(c^{\tau}_R)^2 \right](1-\cos\theta)^2% 
  \Big)
\end{flalign*}

\end {document}



























